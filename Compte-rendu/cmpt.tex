\documentclass[12pt]{article}
\usepackage[french]{babel}
\usepackage[T1]{fontenc}
\usepackage{graphicx} % Required for inserting images
\usepackage[letterpaper,margin=3cm]{geometry}
\usepackage[export]{adjustbox}
\usepackage{listings}
\usepackage{tabularx}
\usepackage{multirow}
\usepackage[table]{xcolor}
\usepackage{svg}
\usepackage{tocloft}
\usepackage{algorithmic}
\usepackage{indentfirst}
\usepackage{hyperref}
\usepackage{array}
\usepackage{float}
\usepackage{caption}
\usepackage{dirtree}

\renewcommand{\thesection}{\Roman{section}} 
\renewcommand\thesubsection{\arabic{subsection}}

\setlength{\cftsecnumwidth}{3em} 
% ajuste la largeur de la colonne des numéros de section

\setlength{\parindent}{1cm}
\sloppy


%--- begin document ------------------------------------------------------------

\title{Projet de raisonnement propositionnel}
\author{Edouard.H Théo.R.V}
\date{2022--2023}

\begin{document}
    \begin{figure}
        \includegraphics[scale=0.3, right]{logo-univ-rouen-normandie-noir.png}
    \end{figure}

    \maketitle

    \begin{abstract}
        Ce document constitue notre compte rendu du projet de web. Il traite 
        de la production d'un site internet de réservation de chambre et de 
        dortoir, conformément à l'énoncé de ce projet.
        
        Le site que nous avons donc produit se nomme \textbf{Taup'Hotel}
        (Jeux de mots avec sa ressemblance avec \textbf{Top Hotel} qui signifie 
        en anglais `le meilleur hôtel'). Nous avons décidé de séparer la 
        production de celui-ci en deux parties. C'est pour cela que nous 
        commenceront par une présentation du `front-end', puis du `back-end'. 
        Pour finir, nous aborderons les difficultés liées à la production de ce 
        projet et des diverses améliorations de ce site. 
    \end{abstract}

    \newpage

    \tableofcontents

    \newpage

    \section{Le front-end}

    Qui par définition selon le site linkweb, correpond à `l'ensemble des 
    éléments visibles et accessibles directement sur un site web'. Pour ce site, 
    cela constitue la production de pages php créant diverse page internet et de
    leurs mises en forme à l'aide des fichiers css. Il s'agit d'ailleur de la 
    première chose que nous avons produit pour ce site.

    \vphantom{}

    Pour la réalisation de ces pages, nous avons tous d'abord réaliser des 
    maquettes nous permettant de mettre en formes toutes ces pages. Après cela 
    nous avons donc réalisé les élement les plus récurrents du site. La 
    production du haut de page ('header') et du bas de page (`footer') a donc 
    été l'étape suivante de cette production. 

    \vphantom{}

    Le `header' contient la redirection vers la page de recherche. En effet, 
    nous avons fait le choix de créer une page aillant pour bute la recherche 
    d'annonce. Contrairement à ce qui était demandé, de l'incorporer dans la page 
    d'acceuile. Au contraire, nous avons créé une page d'acceuille présantant 
    le site à l'aide d'avis fictife. 

    \vphantom{}

    La partie la plus intéréssante de cette page d'acceuille concerne la 
    carousel. Un simple jeu sur des modulos et l'utilisation de la fonction 
    \textbf{setInterval} (fonction qui permet de répéter l'appelle à une 
    procédure spécifier dans un interval donner) permet sa production. 

    \vphantom{}

    Le `header' donne aussi accés au page de connection et d'inscription (dans 
    le cas où l'utilisateur n'est pas connecter). Ces deux pages on recourt au 
    même fichier de style css. Une spécificiter intéréssante de ces pages 
    concerne l'annimation css qui se déclanche en cas d'erreur sur un des champs 
    du formulaire. Celle-ci à recours au `keyframe' élement presque obligatoire 
    à la production d'animation en css. Nous avons aussi eu de petit problème en 
    ce qui concerne cette page, le fond étant une image (libre bien évidement) 
    en résolution 4k, on peut alors observer un fond blanc le temps de 
    chargement de cette image.

    \vphantom{}
    
    Dans le cas où l'utilisateur est connecté, le `header' donne accés à la page 
    de compte. Cette page à plusieur version en fonction du type d'utilisateur 
    qui accède à cette page. Dans le cas d'un utilisateur basique, il a la 
    possibiliter de changer les informations de son compte, mais aussi de noter 
    les chambres et dortoire que l'utilisateur à déjà visiter. Pour un 
    propirétaire de chambre, il a accès à la modifictions des informations de 
    ces annonces, mais aussi à leurs supprésion. Enfin pour un administrateur,
    l'accès à la supprésion d'annonce et d'utilisateur est donné dans cette 
    page. On notera par contre, l'imposibiliter pour un administrateur de 
    réserver, de détenire des chambres. Une telle condition peut être justifiée
    par une limitation pour les administrateurs à influencer les choix des 
    utilisateurs vers des chambres qu'il détiendrait. Enfin cette page donne 
    aussi accès au divers `sessions' du compte, point qui sera dévelloper plus 
    profondément dans la partie sur le `back-end'.

    \vphantom{}

    En ce qui concerne le `footer', celui-ci donne accès à la page `post', qui 
    permet d'ajouter une annonce. En ce qui concerne ce procésusse d'ajout 
    d'annonce, nous avons décidé de rendre possible une telle action seulement
    pour les personnes aillants reçu le grade adéquoite. Grade qui est 
    demandable à un administrateur à l'aide de la page de compte. Cette 
    contrainte sur l'utilisateur pour l'ajout d'annonce (de chambre ou d'hotel) 
    permetterai de limité le nombre d'annonces à administrer pour les 
    administrateurs du site.
    
    \vphantom{}
    
    En ce qui concerne cette page (la page d'ajout d'annonce), la partie la plus 
    intéréssante réside dans la production d'une prévisualisation d'image que 
    l'utilisateur rentrer pour cette annonce. Une telle fonctionnaliter permet 
    pour l'utilisateur de ne pas envoyer de photo possibilement comprométente à 
    notre serveur. En ce qui concerne l'implémentation de cette fonctionnaliter, 
    une simple script js permet ce rendu. De plus, pour cette page, on note la 
    ressemblance avec la page des annonces et/ou hotel. 

    \vphantom{}

    Pour finir on peut évoquer le mode de fonctionnement d'ajout des annonces. 
    Une annonce et soit un hotel, soit une chambre. Un hotel peut contenir 
    plusieur chambre et une chambre doit provenir d'un hotel. Enfin l'ajout 
    d'une chambre à une hotel se fait lors de la création de la chambre. 
    En effet, lors de cette création, l'utilisateur se voit forcer de lier la 
    chambre qu'il vient de créer à un hotel.

    \vphantom{}
    
    Pour conclure cette section sur le `front-end' le choix que nous avons fait 
    de séparer le `front du back' nous a permis une implémentation plus rapide 
    de toutes ces pages php et de leurs styles. Les avantages pour le `back' 
    sont traité dans la deuxième section de ce compte rendu. 

    \newpage

    \section{Back-end}

    Dans cette partie seront évoquer la justification et production de l'API de 
    notre site, mais aussi des liens permettant de lier le `back au front'.

    \subsection{Justification du choix de production d'une API.}

    Le choix de la production d'une API, premièrement permet une séparation 
    concrète entre le coter serveur et celui client. On comprend donc qui 
    exécute quoi. Secondement, un autre `front' pourrait être simplement 
    implémenté. Il ne resterait alors qu'à reproduire les requêtes API semblables. 
    On peut aussi envisager la possibiliter de créer une application mobile ou 
    encore un appilication desktop. 
    Nous allons donc poursuivre cette section par une présentation des 
    fonctionnaliter de cette API.

    \subsection{Gestion globale de l'API.}

    Les requêtes faites à l'API sont toutes dans notre cas fait à l'aide de 
    jquery de la commande \textbf{post}. Toutes requêtes sont traitées par l'API 
    avec un module que nous avons élaboré \textbf{ratelimite}. Ce module permet 
    de s'assurer qu'aucun utilisateur n'est malicieux, c'est-à-dire qu'il ne 
    `spam' pas notre API. Dans le cas de requêtes trop nombreuses pour un 
    utilisateur, celui ce vois bloquer l'accer aux requêtes pour une durée de 
    5minute, valeur choisir arbitrairement.

    \subsection{Gestion de compte}

    La gestion des comptes prévue par notre API peut être vue comme très poussé. 
    En effet, nous avons l'inscription et la connextion avec pour tous les 
    champs concernnet, une vérification de la vérasiter de ces champs, une 
    gestion des mots de passe en chiffré à l'aide de la fonction de hachage 
    \textbf{sha512}. Mais nous avons surtout la gestion de session de compte. À 
    la manière de grand réseaux sociaux comme tiktok ou même de moteur de 
    recherche comme Google, notre site implémente cette fonctionnaliter. 
    Celle-ci corresponds à la création d'une session à chaque nouvelle connexion.
    Une nouvelle connection est définie comme étant les donners d'une session 
    n'existant pas encore. Le principe de ces sessions permet une grande 
    amélioration de la sécuriter des comptes.

    \subsection{La transmition de donner}

    Comme evoquer plus tôt, la plupart de nos pages, éffféctue des requêtes pour 
    accéder au donné qu'il les intéresse. Mais ce mode de fonction ne permet 
    pas la gestion de page comme celle de reservation de chambre. C'est pour 
    cela que nous sommes passé par la fonction \textbf{urlParams}.

    \subsection{La gestion des images d'annonce}

    Les hotels et chambres étant obligatoirement représenté par une image, il 
    nous a donc fallu trouver une méthode pour cette sauvegarde. Nous avons 
    donc décidé d'utiliser l'API de imgure (site de partage de photo) pour la 
    gestion de ces images. Cela permet de réduire l'espace néssécaire au serveur 
    pour la gestion de ce site web en cas de déploiment. Cela pourrait aussi 
    permettre à des utilisateurs de imgure d'être améner à réserver une chambre 
    après l'avoir consulté sur le site. 

    \subsection{Le scroll `infinie' de la page de recherche}

    La page de recherche d'annonce pouvant contenir des dixaines de milier 
    d'annonce, une simple requêtes récupérent toutes ces annonces pourrait être 
    très longues à recevoir et même à traiter. C'est pour cela que nous avons 
    mis en place un scroll qui une fois arrivé au nombre d'annonces actuellement 
    affiché re-éfféctue des requêtes et donc affiche ces nouvelles annonces. Le 
    `ratelimite' de maximal de telles requêtes revient donc à une recherche
    maximale d'envireons 1200 annonce toutes les 5 minutes, ce qui est plutôt r
    raisonble selon nous. 

    \newpage

    \section{Conclusion}

    Pour conclure, ce projet étant asser ouvert, nous avons donc pu inonver par 
    rapport au projet que nous avions produit l'anné dernière notament avec la 
    production d'une API, ou encore le stockage des images sur imgure. En ce qui 
    concercerne les difficultés recontrès, premièrement la production de ce 
    compte rendu. N'aillant pas d'exepérience dans ce domaine, nous ne savions 
    pas de quoi parler ni même comment. Secondement, le temps, nous à fait 
    defaut, plusieur fonctionnaliter et même pages n'ont pas pu être produite 
    à cause de manque de temps. On peut notament citer l'idée de rendre le site 
    responsive (c'est-à-dire, produire des versions du `front' en redimentionant 
    certaine partie du site pour le rendre plus accésible sur des appareils
    comme les téléphones). Il s'agit de la plus du plus grand inconvéniant de ce 
    site, en effet, si l'utilisateur n'a pas une résolution générique de 
    dimension 16:9 alors une perte de silibiliter peut apparaitre, cela rend 
    même le site dans certain cas ilisible.   

    \section{Remerciment}

    Louis Dumontier, pour nous avoir produit la taupe qui nous sert de logo.
    
\end{document}