\documentclass[12pt]{article}
\usepackage[french]{babel}
\usepackage[T1]{fontenc}
\usepackage{graphicx} % Required for inserting images
\usepackage[letterpaper,margin=3cm]{geometry}
\usepackage[export]{adjustbox}
\usepackage{listings}
\usepackage{tabularx}
\usepackage{multirow}
\usepackage[table]{xcolor}
\usepackage{svg}
\usepackage{tocloft}
\usepackage{algorithmic}
\usepackage{indentfirst}
\usepackage{hyperref}
\usepackage{array}
\usepackage{float}
\usepackage{caption}
\usepackage{dirtree}

\renewcommand{\thesection}{\Roman{section}} 
\renewcommand\thesubsection{\arabic{subsection}}

\setlength{\cftsecnumwidth}{3em} 
% ajuste la largeur de la colonne des numéros de section

\setlength{\parindent}{1cm}
\sloppy


%--- begin document ------------------------------------------------------------

\title{Projet de raisonnement propositionnel}
\author{Edouard.H Théo.R.V}
\date{2022--2023}

\begin{document}
    \begin{figure}
        \includegraphics[scale=0.3, right]{logo-univ-rouen-normandie-noir.png}
    \end{figure}

    \maketitle

    \begin{abstract}
        Ce document constitue notre compte rendu du projet de web. Il traite 
        de la production d'un site internet de réservation de chambre et de 
        dortoir, conformément à l'énoncé de ce projet.
        
        Le site que nous avons donc produit se nomme \textbf{Taup'Hotel}
        (Jeux de mots avec sa ressemblance avec \textbf{Top Hotel} qui signifie 
        en anglais `le meilleur hôtel'). Nous avons décidé de séparer la 
        production de celui-ci en deux parties. C'est pour cela que nous 
        commenceront par une présentation du `front-end', puis du `back-end'. 
        Pour finir, nous aborderons les difficultés liées à la production de ce 
        projet et des diverses améliorations de ce site. 
    \end{abstract}

    \newpage

    \tableofcontents

    \newpage

    \section{Le front-end}

    Qui par définition selon le site linkweb, correpond à `l'ensemble des 
    éléments visibles et accessibles directement sur un site web'. Pour ce site, 
    cela constitue la production de pages php créant diverse page internet et de
    leurs mises en forme à l'aide des fichiers css. Il s'agit d'ailleur de la 
    première chose que nous avons produit pour ce site.

    Pour la réalisation de ces pages, nous avons tous d'abord réaliser des 
    maquettes nous permettant de mettre en formes toutes ces pages. Après cela 
    nous avons donc réalisé les élement les plus récurrents du site. La 
    production du haut de page ('header') et du bas de page (`footer') a donc 
    été l'étape suivante de cette production. 

    Le `header' contient la redirection vers la page de recherche. En effet, 
    nous avons fait le choix de créer une page aillant pour bute la recherche 
    d'annonce. Contrairement à ce qui était demandé, de l'incorporer dans la page 
    d'acceuile. Au contraire, nous avons créé une page d'acceuille présantant 
    le site à l'aide d'avis fictife. 

    La partie la plus intéréssante de cette page d'acceuille concerne la 
    carousel. Celui-ci a été réalisé à l'aide d'un script js récupérant des 
    images de taille adéquoite à l'affiche du carousel. Un simple jeu sur des 
    modulos et l'utilisation de la fonction \textbf{setInterval} (fonction qui 
    permet de répéter l'appelle à une procédure spécifier dans un interval 
    donner) permet sa production. 

    Le `header' donne aussi accés au page de connection et d'inscription (dans 
    le cas où l'utilisateur n'est pas connecter). Ces deux pages on recourt au 
    même fichier de style css. Une spécificiter intéréssante de ces pages 
    concerne l'annimation css qui se déclanche en cas d'erreur sur un des champs 
    du formulaire. Celle-ci à recours au `keyframe' élement presque obligatoire 
    à la production d'animation en css. Nous avons aussi eu de petit problème en 
    ce qui concerne cette page, le fond étant une image (libre bien évidement) 
    en résolution 4k, on peut alors observer un fond blanc le temps de 
    chargement de cette image.
    
    Dans le cas où l'utilisateur est connecté, le `header' donne accés à la page 
    de compte. Cette page à plusieur version en fonction du type d'utilisateur 
    qui accède à cette page. Dans le cas d'un utilisateur basique, il a la 
    possibiliter de changer les informations de son compte, mais aussi de noter 
    les chambres et dortoire que l'utilisateur à déjà visiter. Pour un 
    propirétaire de chambre, il a accès à la modifictions des informations de 
    ces annonces, mais aussi à leurs supprésion. Enfin pour un administrateur,
    l'accès à la supprésion d'annonce et d'utilisateur est donné dans cette 
    page. On notera par contre, l'imposibiliter pour un administrateur de 
    réserver, de détenire des chambres. Une telle condition peut être justifiée
    par une limitation pour les administrateurs à influencer les choix des 
    utilisateurs vers des chambres qu'il détiendrait. Enfin cette page donne 
    aussi accès au divers `sessions' du compte, point qui sera dévelloper plus 
    profondément dans la partie sur le `back-end'.

    En ce qui concerne le `footer', celui-ci donne accès à la page `post', qui 
    permet d'ajouter une annonce. En ce qui concerne ce procésusse d'ajout 
    d'annonce, nous avons décidé de rendre possible une telle action seulement
    pour les personnes aillants reçu le grade adéquoite. Grade qui est 
    demandable à un administrateur à l'aide de la page de compte. Cette 
    contrainte sur l'utilisateur pour l'ajout d'annonce (de chambre ou d'hotel) 
    permetterai de limité le nombre d'annonces à administrer pour les 
    administrateurs du site.
    
    En ce qui concerne cette page (la page d'ajout d'annonce), la partie la plus 
    intéréssante réside dans la production d'une prévisualisation d'image que 
    l'utilisateur rentrer pour cette annonce. Une telle fonctionnaliter permet 
    pour l'utilisateur de ne pas envoyer de photo possibilement comprométente à 
    notre serveur. En ce qui concerne l'implémentation de cette fonctionnaliter, 
    une simple script js permet ce rendu. De plus, pour cette page, on note la 
    ressemblance avec la page des annonces et/ou hotel. 

    Pour finir on peut évoquer le mode de fonctionnement d'ajout des annonces. 
    Une annonce et soit un hotel, soit une chambre. Un hotel peut contenir 
    plusieur chambre et une chambre doit provenir d'un hotel. Enfin l'ajout 
    d'une chambre à une hotel se fait lors de la création de la chambre. 
    En effet, lors de cette création, l'utilisateur se voit forcer de lier la 
    chambre qu'il vient de créer à un hotel. 
    
\end{document}